\documentclass{article}
\usepackage{graphicx}
\usepackage{subfigure}
\usepackage{fancyhdr}
\usepackage{extramarks}
\usepackage{amsmath}
\usepackage{amsthm}
\usepackage{amsfonts}
\usepackage{tikz}
\usepackage[plain]{algorithm}
\usepackage{algpseudocode}
\usepackage{fancyvrb}
\usetikzlibrary{automata,positioning}

%
% Basic Document Settings
%

\topmargin=-0.45in
\evensidemargin=0in
\oddsidemargin=0in
\textwidth=6.5in
\textheight=9.0in
\headsep=0.25in

\linespread{1.1}

\pagestyle{fancy}
\lhead{\hmwkAuthorName}
\chead{\hmwkClass\ : \hmwkTitle}
\rhead{\firstxmark}
\lfoot{\lastxmark}
\cfoot{\thepage}

\renewcommand\headrulewidth{0.4pt}
\renewcommand\footrulewidth{0.4pt}

\setlength\parindent{0pt}

%
% Create Problem Sections
%

\newcommand{\enterProblemHeader}[1]{
    \nobreak\extramarks{}{Problem \arabic{#1} continued on next page\ldots}\nobreak{}
    \nobreak\extramarks{Problem \arabic{#1} (continued)}{Problem \arabic{#1} continued on next page\ldots}\nobreak{}
}

\newcommand{\exitProblemHeader}[1]{
    \nobreak\extramarks{Problem \arabic{#1} (continued)}{Problem \arabic{#1} continued on next page\ldots}\nobreak{}
    \stepcounter{#1}
    \nobreak\extramarks{Problem \arabic{#1}}{}\nobreak{}
}

\setcounter{secnumdepth}{0}
\newcounter{partCounter}
\newcounter{homeworkProblemCounter}
\setcounter{homeworkProblemCounter}{1}
\nobreak\extramarks{Problem \arabic{homeworkProblemCounter}}{}\nobreak{}

%
% Homework Problem Environment
%
% This environment takes an optional argument. When given, it will adjust the
% problem counter. This is useful for when the problems given for your
% assignment aren't sequential. See the last 3 problems of this template for an
% example.
%
\newenvironment{homeworkProblem}[1][-1]{
    \ifnum#1>0
        \setcounter{homeworkProblemCounter}{#1}
    \fi
    \section{Problem \arabic{homeworkProblemCounter}}
    \setcounter{partCounter}{1}
    \enterProblemHeader{homeworkProblemCounter}
}{
    \exitProblemHeader{homeworkProblemCounter}
}

%
% Homework Details
%   - Title
%   - Due date
%   - Class
%   - Section/Time
%   - Instructor
%   - Author
%

\newcommand{\hmwkTitle}{Homework\ \#1}
\newcommand{\hmwkDueDate}{\today}
\newcommand{\hmwkClass}{MATH6644 Iterative Methods}
\newcommand{\hmwkClassTime}{}
\newcommand{\hmwkClassInstructor}{}
\newcommand{\hmwkAuthorName}{Komahan Boopathy}

%
% Title Page
%

\title{
    \vspace{2in}
    \textmd{\textbf{\hmwkClass:\ \hmwkTitle}}\\
    \normalsize\vspace{0.1in}\small{\hmwkDueDate}\\
    %\vspace{0.1in}\large{\textit{\hmwkClassInstructor\ \hmwkClassTime}}
    \vspace{3in}
}

\author{\textbf{\Large\hmwkAuthorName}}
\date{}

\renewcommand{\part}[1]{\textbf{\large Part \Alph{partCounter}}\stepcounter{partCounter}\\}

%
% Various Helper Commands
%

% Useful for algorithms
\newcommand{\alg}[1]{\textsc{\bfseries \footnotesize #1}}

% For derivatives
\newcommand{\deriv}[1]{\frac{\mathrm{d}}{\mathrm{d}x} (#1)}

% For partial derivatives
\newcommand{\pderiv}[2]{\frac{\partial #1}{\partial #2}}

% Integral dx
\newcommand{\dx}{\mathrm{d}x}

% Alias for the Solution section header
\newcommand{\solution}{\textbf{\large Solution}}

% Probability commands: Expectation, Variance, Covariance, Bias
\newcommand{\E}{\mathrm{E}}
\newcommand{\Var}{\mathrm{Var}}
\newcommand{\Cov}{\mathrm{Cov}}
\newcommand{\Bias}{\mathrm{Bias}}

\begin{document}
\maketitle
\thispagestyle{empty}

\pagebreak

\begin{homeworkProblem}
  Find an orthonormal basis for the column space of matrix
  $$A = 
  \begin{bmatrix}
    1 & 1 & 0 \\
    1 & 0 & 2 \\
    1 & 0 & 1 \\
    1 & 1 & -1
  \end{bmatrix}$$

  The columns $\mathbf{c_1}$, $\mathbf{c_1}$ and $\mathbf{c_3}$ are
  \underline{not orthonormal} to each other. Gram-Scmidt orthogonalization
  procedure is used to orthogonalize the columns as follows. 
  \\

  The first
  column $\mathbf{c_1}$ is chosen as the reference. 
  $$\mathbf{v_1} = \mathbf{c_1} = 
  \begin{bmatrix}
    1 \\
    1 \\
    1 \\
    1
  \end{bmatrix}$$

  The second vector
  \begin{equation*}
    \begin{split}
    \mathbf{v_2} & = \mathbf{c_2} - \dfrac{(\mathbf{c_2},\mathbf{v_1})}{(\mathbf{v_1},\mathbf{v_1})}\mathbf{v_1} \\
    & =   \begin{bmatrix}
    1 \\
    0 \\
    0 \\
    1
  \end{bmatrix}
    -\dfrac{1}{2}
    \begin{bmatrix}
      1 \\
      1 \\
      1 \\
      1
    \end{bmatrix}
     = 
    \begin{bmatrix}
      1/2 \\
      -1/2 \\
      -1/2 \\
      1/2
    \end{bmatrix}
    \end{split}
  \end{equation*}

  The third vector
    \begin{equation*}
  \begin{split}
      \mathbf{v_3}  & = \mathbf{c_3} - \dfrac{(\mathbf{c_3},\mathbf{v_1})}{(\mathbf{v_1},\mathbf{v_1})}\mathbf{v_1} - \dfrac{(\mathbf{c_3},\mathbf{v_2})}{(\mathbf{v_2},\mathbf{v_2})}\mathbf{v_2} \\
      & =
      \begin{bmatrix}
        0 \\
        2 \\
        1 \\
        -1
      \end{bmatrix}
      -\dfrac{1}{2}
      \begin{bmatrix}
        1 \\
        1 \\
        1 \\
        1
      \end{bmatrix}
      +\dfrac{2}{1}
      \begin{bmatrix}
      1/2 \\
      -1/2 \\
      -1/2 \\
      1/2
      \end{bmatrix}
     = 
     \begin{bmatrix}
       1/2 \\
       1/2 \\
       -1/2 \\
       -1/2
    \end{bmatrix}
    \end{split}
  \end{equation*}

  Finally these vectors are \underline{normalized to unity} and the corresponding
  orthonormal basis for the column space of $\mathbf{A}$ is
  $$\begin{bmatrix}
    1/2 &  1/2 & 1/2  \\
    1/2 & -1/2 & 1/2  \\
    1/2 & -1/2 & -1/2 \\
    1/2 &  1/2 & -1/2
  \end{bmatrix}$$

\end{homeworkProblem}

\begin{homeworkProblem}
  Consider a matrix
  $$A = 
  \begin{bmatrix}
     2 & 2 & -2 \\
     2 & 6 & 0 \\
    -2 & 0 & 7 
  \end{bmatrix}$$
 
  \paragraph{a.} 
  \emph{Is this matrix diagonalizable? Why or why not?}
  \medskip
  A matrix is diagonalizable if there are $n$ linearly independent
  eigenvectors. Then, we can do a similarity transformation of the
  form $J=V^{-1}AV$, where $V$ is the matrix of eigenvectors.  
  \\ 
  
  The
  eigenvalues are found using $|A-\lambda I|=0$ which results in the
  cubic characteristic equation
  $$
  -\lambda^3 + 15\lambda^2 -60\lambda + 32 = 0
  $$ 

  The eigenvalues are $\lambda_1 = 8$, $\lambda_2 =
  \frac{7+\sqrt{33}}{2}$ and $\lambda_2 = \frac{7-\sqrt{33}}{2}$ The
  eigenvalues are distinct (not repeated), therefore we can find
  eigenvectors that form a simularity transformation as shown
  above. Therefore it is possible to diagonalize the matrix.

  \paragraph{b.}
  \emph{Compute the 1-norm, 2-norm, Frobenius norm and infinite-norm.}
  \medskip

  \begin{itemize}
  \item  1-norm of the matrix is the largest column sum i.e. $\max
    \{\sum{c_1}, \sum{c_2}, \sum{c_3}\} = \max\{6,8,9\}=9$

  \item  2-norm of the matrix is the largest singular value (eigenvalue
    in this case) i.e., $\max \{8,\frac{7-\sqrt{33}}{2},\frac{7-\sqrt{33}}{2}\} = 8$.

  \item  Infinite-norm of the matrix is the largest row sum i.e., $\max
    \{\sum{r_1}, \sum{r_2}, \sum{r_3}\} = \max\{8,8,9\}=9$

  \item Frobenius-norm of the matrix is $\sum_{i,j} |A_{ij}|^2 = \sqrt{2^2+2^2+2^2+2^2+6^2+2^2+7^2} = 10.247$.

  \end{itemize}

  \paragraph{c.} 
  \emph{Compute the spectral radius of A. Compare and comment on the values from (b) and (c).}
  \medskip

  The spectral radius of A is $\rho(A) = \max\{\lambda_i\} = 8$.  In
  this case, the spectral radius is same as the 2-norm of the matrix
  A. The 2-norm of the matrix is the smallest norm of the different
  computed norms of the matrix.

\end{homeworkProblem}

\begin{homeworkProblem}
  \emph{Give the matrix expression for the symmetric Gauss-Seidel iterations.}
  \medskip

  The matrix $A = D-E-F$ and the right hand side $b$. The symmetric
  Gauss-Seidel iterations are defined as follows. 
  One iteration of Symmetric GS involves one forward and one reverse
  sweeps, each of ${\cal{O}}(n^2)$ operations.


  \paragraph{Forward Sweep}

  $$(D-E) x^{k+1/2} = F x^{k} + b$$

  The strictly lower triangular system D-E is swept using forward
  substitution method.
  
  \paragraph{Backward Substitution}

  $$(D-F) x^{k+1} = E x^{k+1/2} + b$$

  The strictly upper triangular system D-E is swept using backward
  substitution method.

\end{homeworkProblem}

\begin{homeworkProblem}
Construct a $3 \times 3$ dense matrix with $\rho(A)>1$. Define an
iteration,
$$
\mathbf{x}^{k+1} = G \mathbf{x} + \mathbf{c}.
$$ 
Find an example $\mathbf{x_0}$ and $\mathbf{c}$ such that the
iteration does not converge.

\end{homeworkProblem}

\begin{homeworkProblem}
\emph{Discretize the following differential equation}
  $$ -u^{\prime\prime} + 4 u = 0 $$
  $$ u(0) = -1, u(1) = 2, x\in [0,1]$$ 
\emph{by the central
    difference scheme. Write your linear system of equaitons. Solve
    the system by classical iterations such as Jacobi, Gauss-Seidel,
    SOR with n = 1000. Test the relaxation parameter for several
    values and decide which one is better.  You need to discuss your
    results. }

  \paragraph{a. Discretization of ODE}
  \medskip
  The discretized ODE is  $ - \dfrac{u_{i+1} -2 u_i + u_{i-1}}{h^2} + 4 u_i = 0.$
  This can be rewritten as follows$$- u_{i+1} + (2 + 4h^2) u_i - u_{i-1}  = 0.$$
 
  \paragraph{b. Linear system of equations}
  \medskip
  The matrix 
  $$\mathbf{A} = 
  \begin{bmatrix}
    (2 + 4h^2) & -1         & 0 & 0 \\
    -1         & (2 + 4h^2) & -1 & 0  \\
    0          & -1         & (2 + 4h^2) & -1  \\
    \vdots      &      & \ddots  &   \\
    0          &       0    & -1        & (2 + 4h^2)
  \end{bmatrix}$$
This is a banded tri-diagonal system. The right hand side is
 $$\mathbf{b} = 
  \begin{bmatrix}
    -1 \\
    0 \\
    0 \\
    \vdots\\
    2
  \end{bmatrix}$$
  %\paragraph{c. Relaxation Parameter}
  %\paragraph{d. Iterative solutions}

\end{homeworkProblem}


\end{document}
