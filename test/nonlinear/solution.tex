\documentclass{article}
\usepackage{graphicx}
\usepackage{subfigure}
\usepackage{fancyhdr}
\usepackage{extramarks}
\usepackage{amsmath}
\usepackage{amsthm}
\usepackage{amsfonts}
\usepackage{tikz}
\usepackage[plain]{algorithm}
\usepackage{algpseudocode}
\usepackage{fancyvrb}
\usepackage{multirow}
\usepackage{booktabs}
\usetikzlibrary{automata,positioning}

%
% Basic Document Settings
%

\topmargin=-0.45in
\evensidemargin=0in
\oddsidemargin=0in
\textwidth=6.5in
\textheight=9.0in
\headsep=0.25in

\linespread{1.1}

\pagestyle{fancy}
\lhead{\hmwkAuthorName}
\chead{\hmwkClass\ : \hmwkTitle}
\rhead{\firstxmark}
\lfoot{\lastxmark}
\cfoot{\thepage}

\renewcommand\headrulewidth{0.4pt}
\renewcommand\footrulewidth{0.4pt}

\setlength\parindent{0pt}

%
% Create Problem Sections
%

\newcommand{\enterProblemHeader}[1]{
    \nobreak\extramarks{}{Problem \arabic{#1} continued on next page\ldots}\nobreak{}
    \nobreak\extramarks{Problem \arabic{#1} (continued)}{Problem \arabic{#1} continued on next page\ldots}\nobreak{}
}

\newcommand{\exitProblemHeader}[1]{
    \nobreak\extramarks{Problem \arabic{#1} (continued)}{Problem \arabic{#1} continued on next page\ldots}\nobreak{}
    \stepcounter{#1}
    \nobreak\extramarks{Problem \arabic{#1}}{}\nobreak{}
}

\setcounter{secnumdepth}{0}
\newcounter{partCounter}
\newcounter{homeworkProblemCounter}
\setcounter{homeworkProblemCounter}{1}
\nobreak\extramarks{Problem \arabic{homeworkProblemCounter}}{}\nobreak{}

%
% Homework Problem Environment
%
% This environment takes an optional argument. When given, it will adjust the
% problem counter. This is useful for when the problems given for your
% assignment aren't sequential. See the last 3 problems of this template for an
% example.
%
\newenvironment{homeworkProblem}[1][-1]{
    \ifnum#1>0
        \setcounter{homeworkProblemCounter}{#1}
    \fi
    \section{Problem \arabic{homeworkProblemCounter}}
    \setcounter{partCounter}{1}
    \enterProblemHeader{homeworkProblemCounter}
}{
    \exitProblemHeader{homeworkProblemCounter}
}

%
% Homework Details
%   - Title
%   - Due date
%   - Class
%   - Section/Time
%   - Instructor
%   - Author
%

\newcommand{\hmwkTitle}{Homework\ \# 4}
\newcommand{\hmwkDueDate}{\today}
\newcommand{\hmwkClass}{MATH6644 Iterative Methods}
\newcommand{\hmwkClassTime}{}
\newcommand{\hmwkClassInstructor}{}
\newcommand{\hmwkAuthorName}{Komahan Boopathy}

%
% Title Page
%

\title{
    \vspace{2in}
    \textmd{\textbf{\hmwkClass:\ \hmwkTitle}}\\
    \normalsize\vspace{0.1in}\small{\hmwkDueDate}\\
    %\vspace{0.1in}\large{\textit{\hmwkClassInstructor\ \hmwkClassTime}}
    \vspace{3in}
}

\author{\textbf{\Large\hmwkAuthorName}}
\date{}

\renewcommand{\part}[1]{\textbf{\large Part \Alph{partCounter}}\stepcounter{partCounter}\\}

%
% Various Helper Commands
%

% Useful for algorithms
\newcommand{\alg}[1]{\textsc{\bfseries \footnotesize #1}}

% For derivatives
\newcommand{\deriv}[1]{\frac{\mathrm{d}}{\mathrm{d}x} (#1)}

% For partial derivatives
\newcommand{\pderiv}[2]{\frac{\partial #1}{\partial #2}}

% Integral dx
\newcommand{\dx}{\mathrm{d}x}

% Alias for the Solution section header
\newcommand{\solution}{\textbf{\large Solution}}

% Probability commands: Expectation, Variance, Covariance, Bias
\newcommand{\E}{\mathrm{E}}
\newcommand{\Var}{\mathrm{Var}}
\newcommand{\Cov}{\mathrm{Cov}}
\newcommand{\Bias}{\mathrm{Bias}}

\begin{document}
\maketitle
\thispagestyle{empty}

\pagebreak

\begin{homeworkProblem}

  \emph{Can the performance of Newton iteration be improved by a
    linear change of variables?  That is, for nonsingular $N\times N$
    matrices $A$ and $B$, can the Newton iterates for $F(x)=0$ and
    $AF(Bx)=0$ show any performance difference when started at the same
    initial iterate? What about the chord method?
  }
  
  \medskip

  Yes, its is the question of condition number of the jacobian that
  goes into the Newton iteration.
  
\end{homeworkProblem}

\pagebreak

\begin{homeworkProblem}
  \emph{Assume that the standard assumptions hold, that the cost of a
    function evaluation is ${\cal{O}}(N^2)$ floating-point operations,
    the cost of a Jacobian is ${\cal{O}}(N)$ function evaluations, and
    that $x_0$ is near enough to $x^*$ so that the Newton iteration
    converges quadratically to $x^*$.}
  
  \medskip

  \paragraph{a.}
  \emph{Estimate what is the number of iteration needed to obtain $\|e_n\| \le \epsilon \|e_0\|$, where $\epsilon$ is a small tolerance value.}
  
  At each Newton's iteration $n$ there is:
  \begin{enumerate}
  \item one function evaluation
  \item one Jacobian evaluation
  \end{enumerate}

  \paragraph{b.}
  \emph{What is the number of floating point operations required to get this accuracy?}
  
\end{homeworkProblem}

\pagebreak

\begin{homeworkProblem}

  \emph{Answer the questions in the previous problem for the chord method.}

  \medskip
  
\end{homeworkProblem}

\pagebreak

\begin{homeworkProblem}

  \emph{Write the program that solves single nonlinear equations with
    Newton's method, the chord method, and the secant method. For the
    secant method, use $x_{-1}=0.99x_0$. Apply your program to the
    following function/initial iterate combinations, document and
    explain your results.}
  \begin{enumerate}
  \item $f(x) = 2x^2 -5, \quad x_0=10$
  \item $f(x) = \sin x + x, \quad x_0=0.5$
  \item $f(x) = \cos x, \quad x_0=3$
  \end{enumerate}
  \medskip

\end{homeworkProblem}

\end{document}
