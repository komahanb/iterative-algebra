\documentclass{article}
\usepackage{graphicx}
\usepackage{subfigure}
\usepackage{fancyhdr}
\usepackage{extramarks}
\usepackage{amsmath}
\usepackage{amsthm}
\usepackage{amsfonts}
\usepackage{tikz}
\usepackage[plain]{algorithm}
\usepackage{algpseudocode}
\usepackage{fancyvrb}
\usepackage{multirow}
\usepackage{booktabs}
\usetikzlibrary{automata,positioning}

%
% Basic Document Settings
%

\topmargin=-0.45in
\evensidemargin=0in
\oddsidemargin=0in
\textwidth=6.5in
\textheight=9.0in
\headsep=0.25in

\linespread{1.1}

\pagestyle{fancy}
\lhead{\hmwkAuthorName}
\chead{\hmwkClass\ : \hmwkTitle}
\rhead{\firstxmark}
\lfoot{\lastxmark}
\cfoot{\thepage}

\renewcommand\headrulewidth{0.4pt}
\renewcommand\footrulewidth{0.4pt}

\setlength\parindent{0pt}

%
% Create Problem Sections
%

\newcommand{\enterProblemHeader}[1]{
    \nobreak\extramarks{}{Problem \arabic{#1} continued on next page\ldots}\nobreak{}
    \nobreak\extramarks{Problem \arabic{#1} (continued)}{Problem \arabic{#1} continued on next page\ldots}\nobreak{}
}

\newcommand{\exitProblemHeader}[1]{
    \nobreak\extramarks{Problem \arabic{#1} (continued)}{Problem \arabic{#1} continued on next page\ldots}\nobreak{}
    \stepcounter{#1}
    \nobreak\extramarks{Problem \arabic{#1}}{}\nobreak{}
}

\setcounter{secnumdepth}{0}
\newcounter{partCounter}
\newcounter{homeworkProblemCounter}
\setcounter{homeworkProblemCounter}{1}
\nobreak\extramarks{Problem \arabic{homeworkProblemCounter}}{}\nobreak{}

%
% Homework Problem Environment
%
% This environment takes an optional argument. When given, it will adjust the
% problem counter. This is useful for when the problems given for your
% assignment aren't sequential. See the last 3 problems of this template for an
% example.
%
\newenvironment{homeworkProblem}[1][-1]{
    \ifnum#1>0
        \setcounter{homeworkProblemCounter}{#1}
    \fi
    \section{Problem \arabic{homeworkProblemCounter}}
    \setcounter{partCounter}{1}
    \enterProblemHeader{homeworkProblemCounter}
}{
    \exitProblemHeader{homeworkProblemCounter}
}

%
% Homework Details
%   - Title
%   - Due date
%   - Class
%   - Section/Time
%   - Instructor
%   - Author
%

\newcommand{\hmwkTitle}{Homework\ \# 3}
\newcommand{\hmwkDueDate}{\today}
\newcommand{\hmwkClass}{MATH6644 Iterative Methods}
\newcommand{\hmwkClassTime}{}
\newcommand{\hmwkClassInstructor}{}
\newcommand{\hmwkAuthorName}{Komahan Boopathy}

%
% Title Page
%

\title{
    \vspace{2in}
    \textmd{\textbf{\hmwkClass:\ \hmwkTitle}}\\
    \normalsize\vspace{0.1in}\small{\hmwkDueDate}\\
    %\vspace{0.1in}\large{\textit{\hmwkClassInstructor\ \hmwkClassTime}}
    \vspace{3in}
}

\author{\textbf{\Large\hmwkAuthorName}}
\date{}

\renewcommand{\part}[1]{\textbf{\large Part \Alph{partCounter}}\stepcounter{partCounter}\\}

%
% Various Helper Commands
%

% Useful for algorithms
\newcommand{\alg}[1]{\textsc{\bfseries \footnotesize #1}}

% For derivatives
\newcommand{\deriv}[1]{\frac{\mathrm{d}}{\mathrm{d}x} (#1)}

% For partial derivatives
\newcommand{\pderiv}[2]{\frac{\partial #1}{\partial #2}}

% Integral dx
\newcommand{\dx}{\mathrm{d}x}

% Alias for the Solution section header
\newcommand{\solution}{\textbf{\large Solution}}

% Probability commands: Expectation, Variance, Covariance, Bias
\newcommand{\E}{\mathrm{E}}
\newcommand{\Var}{\mathrm{Var}}
\newcommand{\Cov}{\mathrm{Cov}}
\newcommand{\Bias}{\mathrm{Bias}}

\begin{document}
\maketitle
\thispagestyle{empty}

\pagebreak
\begin{homeworkProblem}

  \emph{Let $A$ be a nonsingular matrix with all singular values in the
    interval $(1 , 2)$.  Estimate the number of CGNR iteration required
    to reduce the relative residual by a factor of $10^{-4}$.}
  
\end{homeworkProblem}

\pagebreak

\begin{homeworkProblem}
  \emph{Carry  out  (without  the  help  of  computers)  the  GMRES  method  to  the  following
    linear system $Ax = b$, where
    $$ A = 
    \begin{bmatrix}
      0 &  0 & 0 & 0 & 1 \\
      1 &  0 & 0 & 0 & 0 \\
      0 &  1 & 0 & 0 & 0 \\
      0 &  0 & 1 & 0 & 0 \\
      0 &  0 & 0 & 1 & 0 \\
    \end{bmatrix},
    b = 
    \begin{bmatrix}
      1 \\
      0 \\
      0 \\
      0 \\
      0 \\
    \end{bmatrix},
    x^{(0)} = 
    \begin{bmatrix}
      0 \\
      0 \\
      0 \\
      0 \\
      0 \\
    \end{bmatrix}.
    $$
  }
\end{homeworkProblem}

\pagebreak

\begin{homeworkProblem}
  \emph{Let $A$ be the skew-symmetric matrix
    $$ A = 
    \begin{bmatrix}
      0 & 1 \\
      -1 & 0 \\
    \end{bmatrix}
    \times I_{n/2},$$
    that is an $n \times n$ block diagonal matrix with $ 2 \times 2$ blocks.
    \begin{enumerate}
    \item Show A is normal
    \item Is A positive definite? Explain your reason.
    \item Can you use CGS to compute the solution of $Ax=b$, where $b$
      is the first unit vector (the first element is one and others are
      zero)? If yes, how many iterations will it be convergent, if not
      why.
    \end{enumerate}
  }
  
\end{homeworkProblem}

\pagebreak

\begin{homeworkProblem}
  \emph{Write your code of CGNE to solve two $n \times n$ linear systems $Ax=b$
    and $By=b$ where
    $$
    A = 
    \begin{bmatrix} 
      0  & 1  &  0  &  \ldots  & 1 \\
      0  & 0  &  1  &  \ldots  & 0 \\
      \vdots  & \vdots & \vdots & \ddots & \vdots \\
      0  & 0  &  0  &  \ldots  & 1 \\
      1  & 0  &  0  &  \ldots  & 0 \\
    \end{bmatrix},
    B = 
    \begin{bmatrix} 
      M_1  &      &     &   &  \\
      & M_2  &     &   &  \\
      &  & \ddots  \\
      &  &  & M_{n/2}
    \end{bmatrix},
    b =
    \begin{bmatrix} 
      1 \\
      0 \\
      \vdots \\
      0 \\
    \end{bmatrix},
    $$ and $x^{(0)} = 0$.  In $M_i$ in $B$ is a $ 2 \times 2$ block
    matrix given by
    $$
    M_j = 
    \begin{bmatrix} 
      1  & j-1 \\
      0  & 1 \\
    \end{bmatrix}.
    $$ Compute the solution with $n = 1000$. Comment on your results.
  }
\end{homeworkProblem}

\end{document}
