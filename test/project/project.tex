\documentclass{article}
\usepackage{graphicx}
\usepackage{subfigure}
\usepackage{fancyhdr}
\usepackage{extramarks}
\usepackage{amsmath}
\usepackage{amsthm}
\usepackage{amsfonts}
\usepackage{tikz}
\usepackage[plain]{algorithm}
\usepackage{algpseudocode}
\usepackage{fancyvrb}
\usepackage{multirow}
\usepackage{booktabs}
\usetikzlibrary{automata,positioning}
\usepackage{subfigure}

%
% Basic Document Settings
%

\topmargin=-0.45in
\evensidemargin=0in
\oddsidemargin=0in
\textwidth=6.5in
\textheight=9.0in
\headsep=0.25in

\linespread{1.1}

\pagestyle{fancy}
\lhead{\hmwkAuthorName}
\chead{\hmwkClass\ : \hmwkTitle}
\rhead{\firstxmark}
\lfoot{\lastxmark}
\cfoot{\thepage}

\renewcommand\headrulewidth{0.4pt}
\renewcommand\footrulewidth{0.4pt}

\setlength\parindent{0pt}

%
% Create Problem Sections
%

\newcommand{\enterProblemHeader}[1]{
    \nobreak\extramarks{}{Problem \arabic{#1} continued on next page\ldots}\nobreak{}
    \nobreak\extramarks{Problem \arabic{#1} (continued)}{Problem \arabic{#1} continued on next page\ldots}\nobreak{}
}

\newcommand{\exitProblemHeader}[1]{
    \nobreak\extramarks{Problem \arabic{#1} (continued)}{Problem \arabic{#1} continued on next page\ldots}\nobreak{}
    \stepcounter{#1}
    \nobreak\extramarks{Problem \arabic{#1}}{}\nobreak{}
}

\setcounter{secnumdepth}{0}
\newcounter{partCounter}
\newcounter{homeworkProblemCounter}
\setcounter{homeworkProblemCounter}{1}
\nobreak\extramarks{Problem \arabic{homeworkProblemCounter}}{}\nobreak{}

%
% Homework Problem Environment
%
% This environment takes an optional argument. When given, it will adjust the
% problem counter. This is useful for when the problems given for your
% assignment aren't sequential. See the last 3 problems of this template for an
% example.
%
\newenvironment{homeworkProblem}[1][-1]{
    \ifnum#1>0
        \setcounter{homeworkProblemCounter}{#1}
    \fi
    \section{Problem \arabic{homeworkProblemCounter}}
    \setcounter{partCounter}{1}
    \enterProblemHeader{homeworkProblemCounter}
}{
    \exitProblemHeader{homeworkProblemCounter}
}

%
% Homework Details
%   - Title
%   - Due date
%   - Class
%   - Section/Time
%   - Instructor
%   - Author
%

\newcommand{\hmwkTitle}{Final Project}
\newcommand{\hmwkDueDate}{\today}
\newcommand{\hmwkClass}{MATH6644 Iterative Methods}
\newcommand{\hmwkClassTime}{}
\newcommand{\hmwkClassInstructor}{}
\newcommand{\hmwkAuthorName}{Komahan Boopathy}

%
% Title Page
%

\title{
    \vspace{2in}
    \textmd{\textbf{\hmwkClass:\ \hmwkTitle}}\\
    \normalsize\vspace{0.1in}\small{\hmwkDueDate}\\
    %\vspace{0.1in}\large{\textit{\hmwkClassInstructor\ \hmwkClassTime}}
    \vspace{3in}
}

\author{\textbf{\Large\hmwkAuthorName}}
\date{}

\renewcommand{\part}[1]{\textbf{\large Part \Alph{partCounter}}\stepcounter{partCounter}\\}

%
% Various Helper Commands
%

% Useful for algorithms
\newcommand{\alg}[1]{\textsc{\bfseries \footnotesize #1}}

% For derivatives
\newcommand{\deriv}[1]{\frac{\mathrm{d}}{\mathrm{d}x} (#1)}

% For partial derivatives
\newcommand{\pderiv}[2]{\frac{\partial #1}{\partial #2}}

% Integral dx
\newcommand{\dx}{\mathrm{d}x}

% Alias for the Solution section header
\newcommand{\solution}{\textbf{\large Solution}}

% Probability commands: Expectation, Variance, Covariance, Bias
\newcommand{\E}{\mathrm{E}}
\newcommand{\Var}{\mathrm{Var}}
\newcommand{\Cov}{\mathrm{Cov}}
\newcommand{\Bias}{\mathrm{Bias}}

\begin{document}
\maketitle
\thispagestyle{empty}

\pagebreak

\section{Preconditioned Conjugate Gradient}

\begin{abstract}
  We explore the solution of symmetric positive definite
  Toeplitz systems by conjugate gradient methods using two different
  circulant preconditioners.
\end{abstract}

\subsection{Introduction}

Toeplitz matrices arise in signal processing and time-series
analysis. The linear system $Ax =b$ is defined completely by $n$
numbers instead of $n^2$ numbers.

\begin{itemize}
  
\item One of the core ideas of using iterative methods for solving
  Toeplitz system is based on the fact that matrix vector products
  $Av$ can be computed efficiently in ${\cal{O}}(n \log n)$ floating
  point operations via the Fast Fourier Transform algorithm
  (compared to ${\cal{O}}(n^2)$) for regular matrix-vector
  products. Recall that the iterative methods are centered upon on
  being able to $(i)$ compute the norms $(ii)$ compute matrix-vector
  products. The success of any iterative methods resides in these
  two operations.
  
\end{itemize}

A good choice of preconditioner $C$ will accelerate the solution of
iterative algotithms such as Conjugate-Gradient method.  The
convergence rate of the conjugate gradient method is dependent on the
eigenspectrum of $C^{-1}A$. A circulant preconditioner $C$ for
toeplitz system $A$ is such that except for the largest and smallest
eigenvalue, all other eigenvalues are clustered around 1. The
convergence of conjugate gradient method is not solely dependent on
the extreme values unlike ordinary classical iteration methods. This
greatly accelerates the convergence of conjugate gradient method for
Toeplitz system with circulant preconditoners.

\begin{itemize}
\item Circulant preconditioners can be diagonalized (which
  effectively means that they are solved) using FFT in ${\cal{O}}(n
  \log n)$ operations.
\end{itemize}

%  A matrix $C$ is said to diagonalize a matrix $A$ if $$ D = C^{-1} A
%  C $$

\subsection{Preconditioners}

We explore the suitability of two preconditioners for Topelitz
systems.

\subsubsection{G. Strang's Circulant Preconditioner}

The form of GS preditioner is $$ C = $$

\subsubsection{T. Chan's Circulant Preconditioner}

The form of TC preditioner is $$ C = $$

\subsection{Summary}

T. Chan's new preconditioner is
easy to compute and performs better than G. Strang's preconditioner
in terms of reducing the condition number of $C^{-1}A$ and
comparably in terms of clustering the eigenspectrum around unity.

%\end{homeworkProblem}

\clearpage

\begin{homeworkProblem}

  Consider the discrete Chandrasekhar H-equation

  $$
  \mathbf{F_i(x)} = x_i - \left( 1 - \frac{c}{2N} \sum_{j=1}^N \dfrac{\mu_i x_j}{\mu_i + \mu_j} \right)
  $$
  
  where $c \in (0,1)$ is a given constant, $\mu_i=(i-1/2)/N$ for $1
  \le i \le N$ where N is the dimension of the unknown vector
  $\mathbf{x}$.
  Compute the solution of the equation using:
  \begin{enumerate}
  \item Fixed Point Method
  \item Chord Method
  \item Newton Method
  \item Shamanskii Method 
  \end{enumerate}

  In all the computations, the initial guesss is taken as $\mathbf{x}=[1,1,1,\ldots,1]^T$, the
  stopping condition is that
  $$
  ||\mathbf{F(x)}|| \le \tau_r r_0 + tau_a,
  $$ where $\tau_r = \tau_a = 10^{-6}$.  The Jacobian matrix needed
  for Chord, Newton and Shamanskii methods are evaluated numerically
  using finite differences with a perturbation size of
  $h=10^{-8}$.

  The error reduction achieved for different method is:
\begin{verbatim}
 Chandrasekhar Equation using Newton method
           1   9.9378833628837296E-006
           2   1.7516420407588525E-013
           3   9.6786999382662530E-016
 Chandrasekhar Equation using chord method
           1   1.6062975875801665E-010
           2   1.6616296724220897E-015
           3   0.0000000000000000     
 Chandrasekhar Equation using fixed point method
           1   1.2124549591147763E-005
           2   1.3670891817549066E-008
           3   1.5414410512575431E-011
           4   1.7468292958981500E-014
           5   2.2204460492503131E-016
 Chandrasekhar Equation using shamanskii method
         chord   1   1.6062975875801665E-010
         chord   2   1.6616296724220897E-015
         chord   3   0.0000000000000000     
         newton  1   0.0000000000000000  
\end{verbatim}

The number of iterations taken by each method to solve
this equation are tabulated in Table~\ref{tab:prob2_iterations}.

\begin{table}[H]
  \caption{Number of iterations for solving Chandrasekhar H-equation.}
  \medskip
  \centering
  \begin{tabular}{c|c|cc}
    \toprule
    No. & Method & Iterations \\
    \midrule
    1 & Fixed point &  \\
    2 & Chord       &  \\
    3 & Newton      &  \\
    4 & Shamanskii  &  \\
    \bottomrule
  \end{tabular}
  \label{tab:prob2_iterations}
\end{table}

\begin{table}[H]
  \caption{Number of iterations for solving Chandrasekhar H-equation.}
  \medskip
  \centering
  \begin{tabular}{c|c|cccc|c}
    \toprule
    No. & Method & \# F  & \# Jacobian & \# Factor & \# ApplyFactor & \# Total \\
    \midrule
    1 & Fixed point &&&&& \\
    2 & Chord       &&&&& \\
    3 & Newton      &&&&&  \\
    4 & Shamanskii  &&&&&  \\
    \bottomrule
  \end{tabular}
  \label{tab:prob2_cost}
\end{table}

Figure~\ref{} shows the plot of solution obtained using these four
methods.
\begin{figure}[H]
  \centering
  \includegraphics[width=0.6\textwidth]{dummy}
  \caption{Solution of Chandrasekhar H-equation obtained using different solution method.}
  \label{fig:chandra_solution}
\end{figure}


\end{homeworkProblem}

\end{document}

\emph{Write the program that solves single nonlinear equations with
  Newton's method, the chord method, and the secant method. For the
  secant method, use $x_{-1}=0.99x_0$. Apply your program to the
  following function/initial iterate combinations, document and
  explain your results.}
\begin{enumerate}
\item $f(x) = 2x^2 -5, \quad x_0=10$
\item $f(x) = \sin x + x, \quad x_0=0.5$
\item $f(x) = \cos x, \quad x_0=3$
\end{enumerate}
\medskip

\begin{table}[H]
  \caption{Number of iterations taken to for the test case.} \medskip \centering
  \begin{tabular}{c|c|ccc}
    \toprule
    Case & Test/Method         & Newton & Chord & Secant \\
    \midrule
    a & $f(x) = 2x^2 -5$    & 6 & 8  & 64  \\\midrule
    b & $f(x) = \sin x + x$ & 3 & 3  & --   \\\midrule
    c & $f(x) = \cos x$     & 4 & --  & -- \\
    \bottomrule
  \end{tabular}
  \label{tab:number_iterations}
\end{table}
The stopping criterion used in $\tau_r=\tau_a=10^{-6}$.

\begin{figure}[H]
  \centering
  \begin{minipage}{0.45\linewidth}
    \includegraphics[width=\textwidth]{f1.pdf}
  \end{minipage}
  \begin{minipage}{0.45\linewidth}
    \includegraphics[width=\textwidth]{f2.pdf}
  \end{minipage}
  \begin{minipage}{0.45\linewidth}
    \includegraphics[width=\textwidth]{f3.pdf}
  \end{minipage}
  \caption{Convergence history with different methods: case (a) (left), case (b) middle, case (c) right .}
  \label{convergence}
\end{figure}

Figure~\ref{convergence} shows the convergence of the three methods
for these problems.

Secant and chord method do not yield the solution for case
(c). The tolerance is satisfied for secant method (shown in
green), but the solution is not the correct solution. This tells
that the stopping criteria is not robust. For the chord method
(shown in blue), the the iterates keep cycling and they do not
converge at all.

About the convergence rates:
\begin{itemize}
\item The Newton's method is quadratic
\item The secant method is superlinear
\item The chord method is linear
\end{itemize}

\end{homeworkProblem}
